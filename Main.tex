% Options for packages loaded elsewhere
\PassOptionsToPackage{unicode}{hyperref}
\PassOptionsToPackage{hyphens}{url}
\PassOptionsToPackage{dvipsnames,svgnames,x11names}{xcolor}
%
\documentclass[
  letterpaper,
  DIV=11,
  numbers=noendperiod]{scrartcl}

\usepackage{amsmath,amssymb}
\usepackage{iftex}
\ifPDFTeX
  \usepackage[T1]{fontenc}
  \usepackage[utf8]{inputenc}
  \usepackage{textcomp} % provide euro and other symbols
\else % if luatex or xetex
  \usepackage{unicode-math}
  \defaultfontfeatures{Scale=MatchLowercase}
  \defaultfontfeatures[\rmfamily]{Ligatures=TeX,Scale=1}
\fi
\usepackage{lmodern}
\ifPDFTeX\else  
    % xetex/luatex font selection
  \setmainfont[]{Inter}
  \setsansfont[]{Inter}
  \setmathfont[]{Fira Math}
\fi
% Use upquote if available, for straight quotes in verbatim environments
\IfFileExists{upquote.sty}{\usepackage{upquote}}{}
\IfFileExists{microtype.sty}{% use microtype if available
  \usepackage[]{microtype}
  \UseMicrotypeSet[protrusion]{basicmath} % disable protrusion for tt fonts
}{}
\makeatletter
\@ifundefined{KOMAClassName}{% if non-KOMA class
  \IfFileExists{parskip.sty}{%
    \usepackage{parskip}
  }{% else
    \setlength{\parindent}{0pt}
    \setlength{\parskip}{6pt plus 2pt minus 1pt}}
}{% if KOMA class
  \KOMAoptions{parskip=half}}
\makeatother
\usepackage{xcolor}
\setlength{\emergencystretch}{3em} % prevent overfull lines
\setcounter{secnumdepth}{5}
% Make \paragraph and \subparagraph free-standing
\ifx\paragraph\undefined\else
  \let\oldparagraph\paragraph
  \renewcommand{\paragraph}[1]{\oldparagraph{#1}\mbox{}}
\fi
\ifx\subparagraph\undefined\else
  \let\oldsubparagraph\subparagraph
  \renewcommand{\subparagraph}[1]{\oldsubparagraph{#1}\mbox{}}
\fi


\providecommand{\tightlist}{%
  \setlength{\itemsep}{0pt}\setlength{\parskip}{0pt}}\usepackage{longtable,booktabs,array}
\usepackage{calc} % for calculating minipage widths
% Correct order of tables after \paragraph or \subparagraph
\usepackage{etoolbox}
\makeatletter
\patchcmd\longtable{\par}{\if@noskipsec\mbox{}\fi\par}{}{}
\makeatother
% Allow footnotes in longtable head/foot
\IfFileExists{footnotehyper.sty}{\usepackage{footnotehyper}}{\usepackage{footnote}}
\makesavenoteenv{longtable}
\usepackage{graphicx}
\makeatletter
\def\maxwidth{\ifdim\Gin@nat@width>\linewidth\linewidth\else\Gin@nat@width\fi}
\def\maxheight{\ifdim\Gin@nat@height>\textheight\textheight\else\Gin@nat@height\fi}
\makeatother
% Scale images if necessary, so that they will not overflow the page
% margins by default, and it is still possible to overwrite the defaults
% using explicit options in \includegraphics[width, height, ...]{}
\setkeys{Gin}{width=\maxwidth,height=\maxheight,keepaspectratio}
% Set default figure placement to htbp
\makeatletter
\def\fps@figure{htbp}
\makeatother

\usepackage{amsmath, xparse}
\usepackage{fancyvrb, fvextra}
\usepackage{unicode-math}
\usepackage{svg}
\usepackage{multicol}
\usepackage{listings}
\usepackage{systeme}
\usepackage{xifthen}
\DefineVerbatimEnvironment{Highlighting}{Verbatim}{breaklines,commandchars=\\\{\}}
\lstset{basicstyle=\ttfamily\footnotesize,breaklines=true}
\newcommand\rowop[1]{\scriptstyle\smash{\xrightarrow[\vphantom{#1}]{\mkern-4mu#1\mkern-4mu}}}
\DeclareDocumentCommand\converttorows%
{>{\SplitList{,}}m}%
{\ProcessList{#1}{\converttorow}}
\NewDocumentCommand{\converttorow}{m}
{\ifthenelse{\isempty{#1}}{}{\rowop{#1}}\\}

\DeclareDocumentCommand \rowops{m}
{\;
\begin{matrix}
\converttorows {#1}
\end{matrix}
\; }
\KOMAoption{captions}{tableheading}
\makeatletter
\@ifpackageloaded{caption}{}{\usepackage{caption}}
\AtBeginDocument{%
\ifdefined\contentsname
  \renewcommand*\contentsname{Table of contents}
\else
  \newcommand\contentsname{Table of contents}
\fi
\ifdefined\listfigurename
  \renewcommand*\listfigurename{List of Figures}
\else
  \newcommand\listfigurename{List of Figures}
\fi
\ifdefined\listtablename
  \renewcommand*\listtablename{List of Tables}
\else
  \newcommand\listtablename{List of Tables}
\fi
\ifdefined\figurename
  \renewcommand*\figurename{Figure}
\else
  \newcommand\figurename{Figure}
\fi
\ifdefined\tablename
  \renewcommand*\tablename{Table}
\else
  \newcommand\tablename{Table}
\fi
}
\@ifpackageloaded{float}{}{\usepackage{float}}
\floatstyle{ruled}
\@ifundefined{c@chapter}{\newfloat{codelisting}{h}{lop}}{\newfloat{codelisting}{h}{lop}[chapter]}
\floatname{codelisting}{Listing}
\newcommand*\listoflistings{\listof{codelisting}{List of Listings}}
\makeatother
\makeatletter
\makeatother
\makeatletter
\@ifpackageloaded{caption}{}{\usepackage{caption}}
\@ifpackageloaded{subcaption}{}{\usepackage{subcaption}}
\makeatother
\ifLuaTeX
  \usepackage{selnolig}  % disable illegal ligatures
\fi
\usepackage{bookmark}

\IfFileExists{xurl.sty}{\usepackage{xurl}}{} % add URL line breaks if available
\urlstyle{same} % disable monospaced font for URLs
\hypersetup{
  colorlinks=true,
  linkcolor={blue},
  filecolor={Maroon},
  citecolor={Blue},
  urlcolor={Blue},
  pdfcreator={LaTeX via pandoc}}

\author{}
\date{}

\begin{document}

\input{./title.tex}
\newpage

\renewcommand*\contentsname{Table of Contents}
{
\hypersetup{linkcolor=}
\setcounter{tocdepth}{4}
\tableofcontents
}
\newpage{}

\section{Types of Goods (01/08)}\label{types-of-goods-0108}

\subsection{Characteristics of the Four Types of
Goods}\label{characteristics-of-the-four-types-of-goods}

\begin{itemize}
\tightlist
\item
  \textbf{Rivalrous} goods are those that can only be consumed by one
  person at a time.
\item
  \textbf{Non-rivalrous} goods are those that can be consumed by
  multiple people at the same time.
\item
  \textbf{Excludable} goods are those that can be restricted to certain
  people.
\item
  \textbf{Non-excludable} goods are those that cannot be restricted to
  certain people.
\item
  If a public good is overcrowded enough, it can become a common
  resource
\end{itemize}

\subsection{The Four Types of Goods}\label{the-four-types-of-goods}

\begin{longtable}[]{@{}
  >{\raggedright\arraybackslash}p{(\columnwidth - 4\tabcolsep) * \real{0.1707}}
  >{\raggedright\arraybackslash}p{(\columnwidth - 4\tabcolsep) * \real{0.5041}}
  >{\raggedright\arraybackslash}p{(\columnwidth - 4\tabcolsep) * \real{0.3171}}@{}}
\toprule\noalign{}
\begin{minipage}[b]{\linewidth}\raggedright
\end{minipage} & \begin{minipage}[b]{\linewidth}\raggedright
\textbf{Non-rivalrous}
\end{minipage} & \begin{minipage}[b]{\linewidth}\raggedright
\textbf{Rivalrous}
\end{minipage} \\
\midrule\noalign{}
\endhead
\bottomrule\noalign{}
\endlastfoot
\textbf{Non-excludable} & \begin{minipage}[t]{\linewidth}\raggedright
\emph{Public Goods}\\
(e.g.~Sunset, Common Knowledge)\strut
\end{minipage} & \begin{minipage}[t]{\linewidth}\raggedright
\emph{Common-Pool/Common Resources}\\
(e.g.~Irrigation Systems, Libraries)\strut
\end{minipage} \\
\textbf{Excludable} & \begin{minipage}[t]{\linewidth}\raggedright
\emph{(Toll/Club/Artificially Scarce) Goods/Natural monopolies}\\
(e.g.~Day-Care Centers, Country Clubs)\strut
\end{minipage} & \begin{minipage}[t]{\linewidth}\raggedright
\emph{Private Goods}\\
(e.g.~Donuts, Personal Computers)\strut
\end{minipage} \\
\end{longtable}

\newpage{}

\subsection{Examples}\label{examples}

\begin{longtable}[]{@{}
  >{\raggedright\arraybackslash}p{(\columnwidth - 2\tabcolsep) * \real{0.7101}}
  >{\raggedright\arraybackslash}p{(\columnwidth - 2\tabcolsep) * \real{0.2899}}@{}}
\toprule\noalign{}
\begin{minipage}[b]{\linewidth}\raggedright
Case Scenario
\end{minipage} & \begin{minipage}[b]{\linewidth}\raggedright
Type of Good/Service
\end{minipage} \\
\midrule\noalign{}
\endhead
\bottomrule\noalign{}
\endlastfoot
A college education & Artificially scarce \\
A manicure or pedicure & Private good \\
Stone Mountain park & Artificially scarce \\
State park campgrounds & Artificially scarce \\
National defense & Public good \\
Peach Pass lane on I-85 & Artificially scarce \\
Fish in the ocean & Common resource \\
Street lights & Public good \\
Netflix/Hulu & Artificially scarce \\
Flu shot & Private good \\
Tornado safety shelter & Public good \\
Bottled water in a tornado safety shelter & Common resource \\
Hearing a tornado siren & Public good \\
Going to an almost empty public beach & Public good \\
Going to an overcrowded public beach & Common resource \\
St.~Lawrence SeaWay & Natural monopoly \\
Flying on a commercial airplane & Natural monopoly \\
Flying a single seat private airplane & Private good \\
Wedding guests eating a slice of the wedding-cake & Common resource \\
Cake sold at a bakery & Private good \\
\end{longtable}

\newpage{}

\section{Introduction to Externalities
(01/09-01/10)}\label{introduction-to-externalities-0109-0110}

\subsection{Overview}\label{overview}

\begin{itemize}
\tightlist
\item
  An \textbf{externality} is a cost/benefit that affects a \emph{third
  party} who did not choose to incur that cost/benefit.
\item
  They are a type of \textbf{market failure} because they are \emph{not}
  accounted for in the price of the good/service.
\item
  The deadweight loss (DWL) of positive externalities will point to the
  right and vice-versa for negative externalities.

  \begin{itemize}
  \tightlist
  \item
    Which means the DWL triangle always points to the social optimum
    quantity.
  \end{itemize}
\end{itemize}

\subsection{\texorpdfstring{Internalizing an Externality (aka \emph{how
to fix an
externality})}{Internalizing an Externality (aka how to fix an externality)}}\label{internalizing-an-externality-aka-how-to-fix-an-externality}

\subsubsection{Problems with
externalities}\label{problems-with-externalities}

\begin{enumerate}
\def\labelenumi{\arabic{enumi})}
\tightlist
\item
  Private individuals won't take into account the external
  costs/benefits
\item
  Public goods and common pool resources tend to lack property rights
\end{enumerate}

\subsubsection{Coase Theorem (the fix!)}\label{coase-theorem-the-fix}

``We can fix externalities without the government if we\ldots{}''

\begin{enumerate}
\def\labelenumi{\arabic{enumi})}
\tightlist
\item
  Give property rights to people
\item
  Minimize transaction costs
\end{enumerate}

\subsubsection{Examples}\label{examples-1}

Methods the government can employ to internalize an externality in a
free market:

\begin{itemize}
\tightlist
\item
  Pollution or emission limits
\item
  ``Pollution credits'' for private firms to buy and sell in the market
\end{itemize}

\newpage{}

\subsection{Positive Externality in
Consumption}\label{positive-externality-in-consumption}

\begin{figure}[H]

{\centering \includegraphics[width=0.5\textwidth,height=\textheight]{img/pos-cons.png}

}

\caption{Positive Externality in Consumption}

\end{figure}%

\subsubsection{Examples}\label{examples-2}

\begin{itemize}
\tightlist
\item
  Consumption of education
\item
  Consumption of health care
\item
  Advertisement can lead to an increase of demand in the free market
  \(\therefore MPB\) goes up and moves the market toward \(MSB\).
\end{itemize}

\subsubsection{Spillover Effect}\label{spillover-effect}

\begin{itemize}
\tightlist
\item
  The spillover effect is \(MEB = MSB-MPB\).
\item
  \(MPB < MSB\)
\item
  \(MPC = MSC\)
\end{itemize}

\subsubsection{Internalizing the Spillover
Effect}\label{internalizing-the-spillover-effect}

\begin{itemize}
\tightlist
\item
  The external \textbf{benefits} can be internalized by
  \textbf{subsidizing} the product/service to the consumers of the
  good/service.
\item
  The government intervention will move the private market to
  \textbf{social optimum} where \(MSB = MSC\).
\end{itemize}

\newpage{}

\subsection{Negative Externality in
Consumption}\label{negative-externality-in-consumption}

\begin{figure}[H]

{\centering \includegraphics[width=0.45\textwidth,height=\textheight]{img/neg-cons.png}

}

\caption{Negative Externality in Consumption}

\end{figure}%

\subsubsection{Examples}\label{examples-3}

\begin{itemize}
\tightlist
\item
  Smoking in public/passive smoking
\item
  Pollution due to fossil fuels
\item
  Playing loud music
\item
  Discarding garbage in public places
\end{itemize}

\subsubsection{Spillover Effect}\label{spillover-effect-1}

\begin{itemize}
\tightlist
\item
  The spillover effect is \(MEB = MSB-MPB\).
\item
  \(MPB > MSB\)
\item
  \(MPC = MSC\)
\end{itemize}

\subsubsection{Internalizing the Spillover
Effect}\label{internalizing-the-spillover-effect-1}

\begin{itemize}
\tightlist
\item
  The external \textbf{benefits} can be internalized by \textbf{imposing
  a tax} on the product/service to the consumers of the good/service.
\item
  The government intervention will move the private market to
  \textbf{social optimum} where \(MSB = MSC\).
\end{itemize}

\newpage{}

\subsection{Positive Externality in
Production}\label{positive-externality-in-production}

\begin{figure}[H]

{\centering \includegraphics[width=0.5\textwidth,height=\textheight]{img/pos-prod.png}

}

\caption{Positive Externality in Production}

\end{figure}%

\subsubsection{Examples}\label{examples-4}

\begin{itemize}
\tightlist
\item
  Companies invest in training/professional development of their
  employees.
\item
  Firms invest in research and development (R\&D).
\end{itemize}

\subsubsection{Spillover Effect}\label{spillover-effect-2}

\begin{itemize}
\tightlist
\item
  The spillover effect is \(MEC = MSC-MPC\).
\item
  \(MPB = MSB\)
\item
  \(MPC > MSC\)
\end{itemize}

\subsubsection{Internalizing the Spillover
Effect}\label{internalizing-the-spillover-effect-2}

\begin{itemize}
\tightlist
\item
  The external \textbf{costs} can be internalized by
  \textbf{subsidizing} the product/service to the producers of the
  good/service.
\item
  The government intervention will move the private market to
  \textbf{social optimum} where \(MSB = MSC\).
\end{itemize}

\newpage{}

\subsection{Negative Externality in
Production}\label{negative-externality-in-production}

\begin{figure}[H]

{\centering \includegraphics[width=0.5\textwidth,height=\textheight]{img/neg-prod.png}

}

\caption{Negative Externality in Production}

\end{figure}%

\subsubsection{Examples}\label{examples-5}

\begin{itemize}
\tightlist
\item
  Firms produce chemicals that cause pollution \(\therefore\) local
  fisherman cannot catch fish.
\item
  Construction of roads lead to change of landscape and parks
\item
  Coal fired power plants
\end{itemize}

\subsubsection{Spillover Effect}\label{spillover-effect-3}

\begin{itemize}
\tightlist
\item
  The spillover effect is \(MEC = MSC-MPC\).
\item
  \(MPB = MSB\)
\item
  \(MPC < MSC\)
\end{itemize}

\subsubsection{Internalizing the Spillover
Effect}\label{internalizing-the-spillover-effect-3}

\begin{itemize}
\tightlist
\item
  The external \textbf{costs} can be internalized by \textbf{imposing a
  tax} on the product/service to the producers of the good/service.
\item
  The government intervention will move the private market to
  \textbf{social optimum} where \(MSB = MSC\).
\end{itemize}

\newpage{}

\section{Income Inequality (01/12)}\label{income-inequality-0112}

\subsection{The Lorenz Curve and Gini
Coefficient}\label{the-lorenz-curve-and-gini-coefficient}

\begin{itemize}
\tightlist
\item
  The \textbf{Lorenz Curve} \(L(x)\) is a graphical representation of
  the distribution of income in a country.

  \begin{itemize}
  \tightlist
  \item
    The x-axis is the cumulative percentage of the population
    (0\%-100\%).
  \item
    The y-axis is the cumulative percentage of income (0\%-100\%).
  \item
    It is always accompanied by the line \(y=x\) which represents
    \textbf{perfect equality}.
  \end{itemize}
\item
  The \textbf{Gini Coefficient} \(G\) is a numerical representation of
  the Lorenz Curve.

  \begin{itemize}
  \tightlist
  \item
    It is the ratio of the area between the Lorenz Curve and the line
    \(y=x\) to the area under the line \(y=x\).

    \begin{itemize}
    \tightlist
    \item
      \(G = \frac{A}{A+B}\) where
      \(A = \int_{0}^{1} \left[x-L(x)\right] \ \mathrm{d}x\) and
      \(B = \int_{0}^{1}L(x) \ \mathrm{d}x\).
    \end{itemize}
  \item
    The closer \(G\) is to 1, the more unequal the distribution of
    income is.
  \end{itemize}
\end{itemize}

\begin{figure}[H]

{\centering \includegraphics[width=0.68\textwidth,height=\textheight]{img/lorenz.png}

}

\caption{Visual depiction of the Lorenz Curve}

\end{figure}%

As demonstrated in \emph{Figure 6} below:

\begin{itemize}
\tightlist
\item
  If \(G\) is 0, then the Lorenz Curve is \textbf{also} the line \(y=x\)
  because the area between both curves \(A\) is 0.
\item
  If \(G\) is 1, then the Lorenz Curve is the x-axis (\(y=0\)) because
  \(A+B\) must also equal the area under \(y=x\), or \(\frac{1}{2}\).
\end{itemize}

\begin{figure}[H]

{\centering \includegraphics[width=1\textwidth,height=\textheight]{img/varying-gini.png}

}

\caption{Varying Gini Coefficients and their corresponding Lorenz
Curves}

\end{figure}%

\subsection{Deriving Simpler Expressions for the Gini
Coefficient}\label{deriving-simpler-expressions-for-the-gini-coefficient}

Since we know that
\(A+B = \int_{0}^{1} x \ \mathrm{d}x = \left.\frac{x^2}{2}\right\vert_{0}^{1} = \frac{1}{2}\),
we can derive ``easier'' expressions to calculate the Gini Coefficient
\(G\).

\subsubsection{\texorpdfstring{Deriving
\(G=2A\)}{Deriving G=2A}}\label{deriving-g2a}

\begin{align}
G &= \frac{A}{A+B}\tag{Initial Gini Coefficient formula} \\
\frac{1}{G} &= \frac{A+B}{A}\tag{Reciprocate} \\
\frac{A}{G} &= A+B\tag{Multiply by $A$} \\
\frac{A}{G}-A &= B\tag{Subtract $A$}
\end{align}

\newpage{}

Now we can substitute \(B\) into the original area formula:

\begin{align}
A + B &= \frac{1}{2}\tag{Area under $y=x$} \\
A+\left(\frac{A}{G}-A\right) &= \frac{1}{2}\tag{Substitute $B$} \\
\frac{A}{G} &= \frac{1}{2}\tag{Simplify} \\
\cfrac{A}{\frac{1}{2}} &= G\tag{Simplify} \\
2A &= G\tag{Multiply by 2}
\end{align}

\subsubsection{\texorpdfstring{Deriving
\(G=1-2B\)}{Deriving G=1-2B}}\label{deriving-g1-2b}

Since we've already expressed \(B\) in terms of \(A\), we just need to
get \(A\) in terms of \(B\).

\begin{align}
G = 2A \tag{Previous derivation} \\
\frac{G}{2} = A \tag{Divide by 2} \\
\frac{G}{2} = \frac{1}{2}-B\tag{Substitute $A$ using the expression $A=\frac{1}{2}-B$} \\
G = 1-2B\tag{Multiply by 2}
\end{align}

Therefore, two \textbf{alternate expressions} for the Gini Coefficient
are:

\begin{align}
G &= 2A \\
G &= 1-2B
\end{align}

\newpage{}

\section{Negative Externalities: Public vs.~Private Resolution and More
on the Coase Theorem
(01/16)}\label{negative-externalities-public-vs.-private-resolution-and-more-on-the-coase-theorem-0116}

\subsection{Conditions}\label{conditions}

Recall that the \textbf{Coase Theorem} states market failures will
always be resolved by the free market. Here are all the conditions for
Coase Theorem to hold true:

\begin{itemize}
\tightlist
\item
  Both sides are rational and willing to negotiate to maximize their own
  utility.
\item
  Low to no transaction costs
\item
  Private property rights are well-defined
\item
  Perfect information is available to both sides and they have the same
  leverage
\end{itemize}

\subsection{Miscellaneous Market
Failures}\label{miscellaneous-market-failures}

There are a couple of other market failures that the government should
try to combat, based on the types of markets we learned about \emph{in
previous units}:

\begin{itemize}
\tightlist
\item
  \textbf{Monopoly}: A single firm controls the entire market.

  \begin{itemize}
  \tightlist
  \item
    This will cause the firm to produce \emph{less} than the social
    optimum and still charge a \emph{greater price}.
  \end{itemize}
\item
  \textbf{Monopsony}: A single firm controls the entire labor market.

  \begin{itemize}
  \tightlist
  \item
    This will cause the firm to hire \emph{less} than optimum and for a
    \emph{lower wage}.
  \end{itemize}
\end{itemize}

\newpage{}

\section{Intro to Macronomic Indicators
(01/22-01/23)}\label{intro-to-macronomic-indicators-0122-0123}

GDP stands for \textbf{Gross Domestic Product.} Those three words are
important to understand:

\begin{itemize}
\tightlist
\item
  Gross: Not just profits - \emph{total} value
\item
  Domestic: Made WITHIN the borders of the US
\item
  Products: Goods and services which have been produced
\end{itemize}

Formal definition of GDP: ~ \textbf{GDP} is: the sum of the market value
of all final goods and services produced within the United States in a
given time period (usually a year).

\begin{longtable}[]{@{}
  >{\raggedright\arraybackslash}p{(\columnwidth - 4\tabcolsep) * \real{0.1143}}
  >{\raggedright\arraybackslash}p{(\columnwidth - 4\tabcolsep) * \real{0.4857}}
  >{\raggedright\arraybackslash}p{(\columnwidth - 4\tabcolsep) * \real{0.4000}}@{}}
\toprule\noalign{}
\begin{minipage}[b]{\linewidth}\raggedright
Concept
\end{minipage} & \begin{minipage}[b]{\linewidth}\raggedright
Is
\end{minipage} & \begin{minipage}[b]{\linewidth}\raggedright
Is NOT
\end{minipage} \\
\midrule\noalign{}
\endhead
\bottomrule\noalign{}
\endlastfoot
Sum & TOTAL & Single industry or market \\
Value & Based on the market price of the goods and services &
subjective \\
Final & new and complete goods and services; ready for use &
intermediate goods/services; used goods \\
G\&S & ONLY a good or service & FINANCIAL ASSETS
(Stock/Bonds/ETFs/Crypto) \\
Produced & MADE & transfer payments or foreign aid \\
Domestically & WITHIN US borders & US citizens abroad \\
Time & This year (NEW production) & OLD G\&S, Goodwill \\
\end{longtable}

\subsection{Calculating GDP}\label{calculating-gdp}

GDP is known as tam easure of \emph{national income accounting}. What
are the two accounting techniques used in measuring GDP?

\begin{itemize}
\tightlist
\item
  \textbf{Expenditure Approach}: Measures GDP by adding up all the
  spending on final goods and services produced in the nation during the
  year.

  \begin{itemize}
  \tightlist
  \item
    \(GDP = C + I + G + (X-M)\); C=Consumption, I=Investment,
    G=Government Spending, X=Exports, M=Imports
  \end{itemize}
\item
  \textbf{Income Approach}: Measures GDP by adding up all the income
  earned by the factors of production (land, labor, capital,
  entrepreneurship) during the year.

  \begin{itemize}
  \tightlist
  \item
    \(GDP = W + I + R + P\); W=Wages, I=Interest, R=Rents, P=Profits
  \end{itemize}
\item
  Therefore, \(GDP = C + I + G + (X-M) = W + I + R + P\)
\end{itemize}

\newpage{}

\subsection{Circular Flow Model and
Leakages}\label{circular-flow-model-and-leakages}

\begin{figure}[H]

{\centering \includegraphics[width=0.75\textwidth,height=\textheight]{img/circular-flow.jpg}

}

\caption{Circular Flow Model}

\end{figure}%

\begin{itemize}
\tightlist
\item
  \textbf{Leakages} are the non-consumption uses of income, such as
  savings, taxes, and imports.
\item
  Your piggy bank and transfer payments are leakages.
\end{itemize}

\subsection{Gross National Product}\label{gross-national-product}

\begin{itemize}
\tightlist
\item
  \textbf{GNP} is the sum of the value of all final goods and services
  prodced by Americans anywhere in the world during a time period.
\end{itemize}

\subsection{Examples of Factors that Affect
GDP}\label{examples-of-factors-that-affect-gdp}

\subsubsection{Which of these is Counted in
GDP?}\label{which-of-these-is-counted-in-gdp}

\begin{itemize}
\tightlist
\item
  A monthly check received by an economics student who has been granted
  a government scholarship \(\times\)
\item
  \textbf{A farmer's purchase of a new tractor}
\item
  A plumber's purchase of a two-year-old used truck \(\times\)
\item
  Cashing a U.S. government bond \(\times\)
\item
  The services of a mechanic in fixing the radiator in his own car
  \(\times\)
\item
  A Social Security check from the government to a retired store clerk
  \(\times\)
\item
  \textbf{An increase in business inventories}
\item
  \textbf{The government's purchase of a new submarine for the Navy}
\item
  \textbf{A barber's income from cutting hair}
\item
  Income received from the sale of Nike Stock \(\times\)
\end{itemize}

\subsubsection{Which of these is counted in GDP and part of
consumption?}\label{which-of-these-is-counted-in-gdp-and-part-of-consumption}

\begin{itemize}
\tightlist
\item
  \textbf{You Spend \$7 at the movies}
\item
  A family pays a contractor \$200k for a house he built them this year
  \(\times\)
\item
  A family pays \$75k for a house built three years ago \(\times\)
\item
  \textbf{An accountant pays a tailor \$175 to sew a suit for her}
\item
  The government increases its defense expenditures by \$1 billion
  \(\times\)
\item
  The government makes a \$300 Social Security payment to a retired
  person \(\times\)
\item
  You buy General Motors Corp.~stock for \$1k in the stock market
  \(\times\)
\item
  At the end of the year, a flour-milling firm finds that its
  inventories of grain and flour are \$10k above the amounts of its
  inventories at the beginning of the year \(\times\)
\item
  A homemaker works hard caring for her spouse and two children
  \(\times\)
\item
  Ford Motor Co.~buys new auto-making robots \(\times\)
\item
  \textbf{You pay \$300 a month to rent an apartment}
\item
  Apple Computers builds a new factory in the US \(\times\)
\item
  RJ Reynolds Co.~buys control of Nabisco \(\times\)
\item
  You buy a new Toyota that was made in Japan \(\times\)
\item
  \textbf{You pay tuition to attend college}
\end{itemize}

\subsubsection{Which of these is Counted in GDP and part of
investment?}\label{which-of-these-is-counted-in-gdp-and-part-of-investment}

\begin{itemize}
\tightlist
\item
  \textbf{A family pays a contractor \$200k for a house he built them
  this year}
\item
  A family pays \$75k for a house built three years ago \(\times\)
\item
  The government increases its defense expenditures by \$1 billion
  \(\times\)
\item
  The government makes a \$300 Social Security payment to a retired
  person \(\times\)
\item
  You buy General Motors Corp.~stock for \$1k in the stock market
  \(\times\)
\item
  \textbf{At the end of the year, a flour-milling firm finds that its
  inventories of grain and flour are \$10k above the amounts of its
  inventories at the beginning of the year}
\item
  A homemaker works hard caring for her spouse and two children
  \(\times\)
\item
  Ford Motor Co.~buys new auto-making robots \(\checkmark\) \(\times\)
\item
  \textbf{Apple Computers builds a new factory in the US}
\item
  RJ Reynolds Co.~buys control of Nabisco \(\times\)
\item
  You buy a new Toyota that was made in Japan \(\times\)
\end{itemize}

\subsubsection{Which of these is Counted in GDP and part of government
spending?}\label{which-of-these-is-counted-in-gdp-and-part-of-government-spending}

\begin{itemize}
\tightlist
\item
  A family pays \$75k for a house built three years ago \(\times\)
\item
  \textbf{The government increases its defense expenditures by \$1
  billion}
\item
  The government makes a \$300 Social Security payment to a retired
  person \(\times\)
\item
  You buy General Motors Corp.~stock for \$1k in the stock market
  \(\times\)
\item
  A homemaker works hard caring for her spouse and two children
  \(\times\)
\item
  RJ Reynolds Co.~buys control of Nabisco \(\times\)
\item
  You buy a new Toyota that was made in Japan \(\times\)
\end{itemize}

\subsubsection{Which of these is Counted in GDP and part of net
export/import?}\label{which-of-these-is-counted-in-gdp-and-part-of-net-exportimport}

\begin{itemize}
\tightlist
\item
  A family pays \$75k for a house built three years ago \(\times\)
\item
  The government makes a \$300 Social Security payment to a retired
  person \(\times\)
\item
  You buy General Motors Corp.~stock for \$1k in the stock market
  \(\times\)
\item
  A homemaker works hard caring for her spouse and two children
  \(\times\)
\item
  RJ Reynolds Co.~buys control of Nabisco \(\times\)
\item
  \textbf{You buy a new Toyota that was made in Japan}
\end{itemize}

\subsubsection{We count only the final price of a good or service in
GDP.
Why?}\label{we-count-only-the-final-price-of-a-good-or-service-in-gdp.-why}

We don't count intermediate and used goods/services because then we
would be \textbf{double-counting}; also, the good/service in question
might have not been made in the time period analyzed if it wasn't final.

\subsubsection{A purely financial transaction will not be counted in
GDP.
Why?}\label{a-purely-financial-transaction-will-not-be-counted-in-gdp.-why}

Because a purely financial transaction doesn't involve consumption,
investment, government spending, exports, or imports.

\subsubsection{When a home-owner does home-improvement work, the labor
is not counted in GDP.
Why?}\label{when-a-home-owner-does-home-improvement-work-the-labor-is-not-counted-in-gdp.-why}

They're not paying themself or making any profits off of their work to
contribute to the income approach for calculating GDP.

\subsection{Calculating GDP Examples (I didn't finish
this)}\label{calculating-gdp-examples-i-didnt-finish-this}

\subsubsection{Example 1}\label{example-1}

Suppose that personal income is \$500 billion, personal taxes are \$100
billion, and depreciation is \$50 billion. Disposable income is equal to
which of the following?

\(DI = PT - PT = 100 - 50 = \$50 \text{ billion}\)

\subsubsection{My Practice}\label{my-practice}

Suppose that personal income is \$100 billion, personal taxes are \$50
billion, and depreciation is \$25 billion. Disposable income is equal to
which of the following?

\subsubsection{Example 2}\label{example-2}

\begin{longtable}[]{@{}
  >{\raggedright\arraybackslash}p{(\columnwidth - 2\tabcolsep) * \real{0.4167}}
  >{\raggedright\arraybackslash}p{(\columnwidth - 2\tabcolsep) * \real{0.2083}}@{}}
\toprule\noalign{}
\endhead
\bottomrule\noalign{}
\endlastfoot
Wages & \$50 Billion \\
Rent & \$20 Billion \\
Private Investment Spending & \$10 Billion \\
Exports & \$30 Billion \\
Interest Payments & \$40 Billion \\
HH Profit & \$80 Billion \\
\end{longtable}

What is the GDP?
\(GDP = W + R + I + X + P = 50 + 20 + 10 + 30 + 80 = \$190 \text{ billion}\)

\subsubsection{My Practice}\label{my-practice-1}

\begin{longtable}[]{@{}
  >{\raggedright\arraybackslash}p{(\columnwidth - 2\tabcolsep) * \real{0.4167}}
  >{\raggedright\arraybackslash}p{(\columnwidth - 2\tabcolsep) * \real{0.2222}}@{}}
\toprule\noalign{}
\endhead
\bottomrule\noalign{}
\endlastfoot
Wages & \$90 Billion \\
Rent & \$40 Billion \\
Private Investment Spending & \$10 Billion \\
Corporate Taxes & \$50 Billion \\
Interest Payments & \$100 Billion \\
HH Profit & \$90 Billion \\
\end{longtable}

What is the GDP?
\(GDP = W + R + I + X + P = 90 + 40 + 10 + 0 + 90 = \$230 \text{ billion}\)

\subsubsection{Example 3}\label{example-3}

\begin{longtable}[]{@{}
  >{\raggedright\arraybackslash}p{(\columnwidth - 2\tabcolsep) * \real{0.4167}}
  >{\raggedright\arraybackslash}p{(\columnwidth - 2\tabcolsep) * \real{0.2083}}@{}}
\toprule\noalign{}
\endhead
\bottomrule\noalign{}
\endlastfoot
Consumption Spending & \$50 Billion \\
Individual Income Taxes & \$20 Billion \\
Private Investment Spending & \$10 Billion \\
Corporate Taxes & \$20 Billion \\
Exports & \$30 Billion \\
Imports & \$40 Billion \\
Government Purchases & \$80 Billion \\
\end{longtable}

What is the GDP?

\subsubsection{My Practice}\label{my-practice-2}

\begin{longtable}[]{@{}
  >{\raggedright\arraybackslash}p{(\columnwidth - 2\tabcolsep) * \real{0.4167}}
  >{\raggedright\arraybackslash}p{(\columnwidth - 2\tabcolsep) * \real{0.2222}}@{}}
\toprule\noalign{}
\endhead
\bottomrule\noalign{}
\endlastfoot
Consumption Spending & \$70 Billion \\
State Income Taxes & \$10 Billion \\
Private Investment Spending & \$50 Billion \\
Corporate Taxes & \$80 Billion \\
Net Exports & -\$40 Billion \\
Government Purchases & \$50 Billion \\
\end{longtable}

What is the GDP?

Using the prior table and the expenditure approach, what percent of GDP
is comprised of consumption, investment, and government spending?

How is this possible?

If a firm experiences depreciation of factor resources, which component
of GDP is negatively affected?

\subsubsection{Example 4}\label{example-4}

\begin{longtable}[]{@{}
  >{\raggedright\arraybackslash}p{(\columnwidth - 2\tabcolsep) * \real{0.3333}}
  >{\raggedright\arraybackslash}p{(\columnwidth - 2\tabcolsep) * \real{0.2639}}@{}}
\toprule\noalign{}
\endhead
\bottomrule\noalign{}
\endlastfoot
Aggregate Data & Value (Billions) \\
Consumption Spending & 10 \\
Employee Compensation & 7 \\
Government Spending & 60 \\
Interest Payments & 10 \\
Net Exports & -50 \\
Profits & 5 \\
Rents & 5 \\
Savings & 10 \\
\end{longtable}

Calculate GDP using both approaches.

Do both approaches yield equal GDP values? Why or why not?

\subsubsection{My Practice}\label{my-practice-3}

\begin{longtable}[]{@{}
  >{\raggedright\arraybackslash}p{(\columnwidth - 2\tabcolsep) * \real{0.3333}}
  >{\raggedright\arraybackslash}p{(\columnwidth - 2\tabcolsep) * \real{0.2639}}@{}}
\toprule\noalign{}
\endhead
\bottomrule\noalign{}
\endlastfoot
Aggregate Data & Value (Billions) \\
Consumption Spending & 190 \\
Employee Compensation & 200 \\
Government Spending & 100 \\
Interest Payments & 100 \\
Investment Spending & 90 \\
Net Exports & 60 \\
Profits & 50 \\
Rents & 50 \\
Savings & 50 \\
\end{longtable}

Calculate GDP using both approaches.

Do both approaches yield equal GDP values? Why or why not?

\subsubsection{Example 5}\label{example-5}

A country consists of 2 firms. Firm A's total revenue is \$200 million.
The cost of their inputs is \$50 million. Firm B's total revenue is
\$100 million. The cost of their inputs is \$10 million. What is the
total value added in this economy?

\subsubsection{My Practice}\label{my-practice-4}

\begin{longtable}[]{@{}
  >{\raggedright\arraybackslash}p{(\columnwidth - 6\tabcolsep) * \real{0.3472}}
  >{\raggedright\arraybackslash}p{(\columnwidth - 6\tabcolsep) * \real{0.1250}}
  >{\raggedright\arraybackslash}p{(\columnwidth - 6\tabcolsep) * \real{0.1250}}
  >{\raggedright\arraybackslash}p{(\columnwidth - 6\tabcolsep) * \real{0.1250}}@{}}
\toprule\noalign{}
\endhead
\bottomrule\noalign{}
\endlastfoot
Kingdom of Burbonia & Firm A & Firm B & Firm C \\
Firm's Sales & 20 & 50 & 100 \\
\begin{minipage}[t]{\linewidth}\raggedright
Cost of Intermediate\\
Goods Purchased by\\
Each firm\strut
\end{minipage} & \begin{minipage}[t]{\linewidth}\raggedright
10\\
\strut \\
\strut
\end{minipage} & \begin{minipage}[t]{\linewidth}\raggedright
40\\
\strut \\
\strut
\end{minipage} & \begin{minipage}[t]{\linewidth}\raggedright
40\\
\strut \\
\strut
\end{minipage} \\
\end{longtable}

What is the total value added in the Kingdom of Burbonia, measured in
millions of dollars?

\subsubsection{Challenge Problem}\label{challenge-problem}

Consumption is one third of total GDP. Gross Private Investment Spending
and Government Spending, Combined, are equal to consumption spending.
Exports are twice the number of imports. Imports are \$50 million.
Government spending is four times as much as investment.

What is consumption spending?

What is investment spending?

What is government spending?

What are exports?

What is GDP?

\newpage{}

\section{Advanced GDP Calculations
(01/25)}\label{advanced-gdp-calculations-0125}

\subsection{NGDP vs.~RGDP}\label{ngdp-vs.-rgdp}

\subsubsection{If we want to measure the amount of production using
current prices, what economic measure should we
use?}\label{if-we-want-to-measure-the-amount-of-production-using-current-prices-what-economic-measure-should-we-use}

Nominal GDP

\subsubsection{If we want to measure the amount of production using base
year prices, what economic measure should we
use?}\label{if-we-want-to-measure-the-amount-of-production-using-base-year-prices-what-economic-measure-should-we-use}

Real GDP

\subsubsection{Define ``REAL''}\label{define-real}

Accounting for inflation by referencing some initial level of price.

\subsubsection{What does RGDP show?}\label{what-does-rgdp-show}

The measure of true product, accounting for inflation

\subsubsection{What are the formulas for types of
GDP?}\label{what-are-the-formulas-for-types-of-gdp}

\begin{longtable}[]{@{}cc@{}}
\toprule\noalign{}
Nominal GDP & Real GDP \\
\midrule\noalign{}
\endhead
\bottomrule\noalign{}
\endlastfoot
\(\Sigma (Q_c \cdot P_c)\) & \(\Sigma (Q_c \cdot P_{c,\text{base}})\) \\
\end{longtable}

\subsubsection{Growth rate (percent change)
formula:}\label{growth-rate-percent-change-formula}

\(\frac{\text{New GDP}-\text{Old GDP}}{\text{Old GDP}} \cdot 100\%\)

For the base year, the RGDP always equals the NGDP.

\subsubsection{Standard of living}\label{standard-of-living}

\begin{itemize}
\tightlist
\item
  We use RGDP to measure the standard of living because it accounts for
  inflation.
\item
  RGDP per capita is the best measure of standard of living.
\end{itemize}

\subsubsection{Inflation/Deflation}\label{inflationdeflation}

\begin{itemize}
\tightlist
\item
  \(\frac{NGDP}{RGDP} \cdot 100\%\) is the deflator value (\(DF\)).
\item
  If \(DF > 100\%\) (\(NGDP > RGDP\)), there is inflation.
\item
  If \(DF < 100\%\) (\(NGDP < RGDP\)), there is deflation.
\item
  If \(DF = 100\%\) (\(NGDP = RGDP\)), prices are staying the same.
\item
  Disinflation is when the rate of inflation is decreasing, but
  inflation is still occurring nonetheless.
\item
  The deflator is always 100\% in the base year.
\end{itemize}

\subsubsection{What limitations does GDP have as an economic
measure?}\label{what-limitations-does-gdp-have-as-an-economic-measure}

\begin{itemize}
\tightlist
\item
  It doesn't account for non-market production (e.g.~stay-at-home
  parents)
\item
  It doesn't account for the underground economy (e.g.~drug dealers)
\item
  It doesn't account for negative externalities (e.g.~pollution)
\end{itemize}

\section{Unemployment (01/29)}\label{unemployment-0129}

\subsection{Questions on introductory unemployment
terms}\label{questions-on-introductory-unemployment-terms}

\begin{enumerate}
\def\labelenumi{\arabic{enumi}.}
\tightlist
\item
  Who constitutes as being employed?
\end{enumerate}

If they worked full or part time during the past week or is on vacation
or sick leave from a regular job.

\begin{enumerate}
\def\labelenumi{\arabic{enumi}.}
\setcounter{enumi}{1}
\tightlist
\item
  Who constitutes as being unemployed?
\end{enumerate}

If they did not work during the preceding week but made some effort to
find work in the past four weeks.

\begin{enumerate}
\def\labelenumi{\arabic{enumi}.}
\setcounter{enumi}{2}
\tightlist
\item
  Who constitutes as being out of the labor force?
\end{enumerate}

People who are not employed and haven't looked for a job in four weeks.
Institutionalized (prison), military, and those younger than 16 as well.
Discouraged workers, full time students, unpaid homemakers, and retirees
are examples.

\begin{enumerate}
\def\labelenumi{\arabic{enumi}.}
\setcounter{enumi}{3}
\tightlist
\item
  Who constitutes a discouraged worker? Do they cause an underestimate
  or overestimate of the unemployment rate?
\end{enumerate}

Workers tha thave given up looking ofr a job, now considered out of the
labor force.

\begin{enumerate}
\def\labelenumi{\arabic{enumi}.}
\setcounter{enumi}{4}
\tightlist
\item
  Is the entire population considered for unemployment calculations?
\end{enumerate}

Only adults (16+), non-institutionalized, civilian, nonretired
population.

\subsection{Types of unemployment}\label{types-of-unemployment}

\textbf{Frictional unemployment}

\begin{itemize}
\tightlist
\item
  Unemployment that is initiated by workers themselves, who are in
  between jobs.
\item
  ``You quit your job and are looking for a new one''
\end{itemize}

\textbf{Structural unemployment}

\begin{itemize}
\tightlist
\item
  When the firm doesn't need the worker anymore
\item
  ``Your skills are no longer needed''
\item
  This could indicate a displacement of workers by technology
\end{itemize}

\textbf{Cyclical unemployment}

\begin{itemize}
\tightlist
\item
  Unemployment that is caused by a recession
\item
  ``Your skills are still needed, but the economy is not doing well''
\end{itemize}

\subsection{Unemployment calculations}\label{unemployment-calculations}

\begin{enumerate}
\def\labelenumi{\arabic{enumi}.}
\tightlist
\item
  Total Population:
\end{enumerate}

18+9+2+1 = 30

\begin{enumerate}
\def\labelenumi{\arabic{enumi}.}
\setcounter{enumi}{1}
\tightlist
\item
  Total Adult Working-Age Population
\end{enumerate}

18+9+2 = 29

\begin{enumerate}
\def\labelenumi{\arabic{enumi}.}
\setcounter{enumi}{2}
\tightlist
\item
  Total Employed
\end{enumerate}

18

\begin{enumerate}
\def\labelenumi{\arabic{enumi}.}
\setcounter{enumi}{3}
\tightlist
\item
  Total Unemployed
\end{enumerate}

9

\begin{enumerate}
\def\labelenumi{\arabic{enumi}.}
\setcounter{enumi}{4}
\tightlist
\item
  Total Labor Force:
\end{enumerate}

\(EM+UE = 18+9=27\)

Finding the three important rates

Note: \(ER + UR = 100\%\)

\begin{enumerate}
\def\labelenumi{\arabic{enumi}.}
\tightlist
\item
  Labor Force Participation Rate (LFPR):
\end{enumerate}

\(\frac{LF}{TAWAP} \cdot 100\% = \frac{27}{29} \cdot 100\% = 93.1\%\)

\begin{enumerate}
\def\labelenumi{\arabic{enumi}.}
\setcounter{enumi}{1}
\tightlist
\item
  Employment Rate (ER)
\end{enumerate}

\(\frac{EM}{LF} = \frac{EM}{EM+UE} \cdot 100\% = \frac{18}{27} \cdot 100\% = 66.7\%\)

\begin{enumerate}
\def\labelenumi{\arabic{enumi}.}
\setcounter{enumi}{2}
\tightlist
\item
  Unemployment Rate (UR)
\end{enumerate}

\(\frac{UE}{LF} = \frac{UE}{(EM+UE)} = \frac{9}{27} = 33.3\%\)

\subsection{The Flow Problem}\label{the-flow-problem}

Day 1: Country of Burbonia has 10 citizens. All citizens are of the
working-age population. 3 are UE. 1 is Discouraged. 6 are EM.

ER: 67.66\%, UR: 33.33\%, LFPR: 90\%

Day 2: One worker loses their job to the machine, but they continue to
look for employment.

\begin{itemize}
\tightlist
\item
  4 UE, 1 Discouraged, 5 EM
\item
  ER: 55.56\%, UR: 44.44\%, LFPR: 90\%
\item
  ER: Decreases, UR: Increases, LFPR: Stays the same
\end{itemize}

Day 3: One unemployed laborer becomes discouraged.

\begin{itemize}
\tightlist
\item
  3 UE, 2 Discouraged, 5 EM
\item
  ER: 62.5\%, UR: 37.5\%, LFPR: 80\%
\item
  ER: Increases, UR: Decreases, LFPR: Decreases
\end{itemize}

Day 4: One discouraged worker starts looking for a job, but there are no
available positions.

\begin{itemize}
\tightlist
\item
  4 UE, 1 Discouraged, 5 EM
\item
  ER: 55.56\%, UR: 44.44\%, LFPR: 90\%
\item
  ER: Decreases, UR: Increases, LFPR: Increases
\end{itemize}

Day 5: Another discouraged worker finds a new job and starts working
immediately.

\begin{itemize}
\tightlist
\item
  4 UE, 0 Discouraged, 6 EM
\item
  ER: 60\%, UR: 40\%, LFPR: 100\%
\item
  ER: Increases, UR: Decreases, LFPR: Increases
\end{itemize}

\begin{longtable}[]{@{}
  >{\centering\arraybackslash}p{(\columnwidth - 12\tabcolsep) * \real{0.1778}}
  >{\centering\arraybackslash}p{(\columnwidth - 12\tabcolsep) * \real{0.1222}}
  >{\centering\arraybackslash}p{(\columnwidth - 12\tabcolsep) * \real{0.1444}}
  >{\centering\arraybackslash}p{(\columnwidth - 12\tabcolsep) * \real{0.1444}}
  >{\centering\arraybackslash}p{(\columnwidth - 12\tabcolsep) * \real{0.1222}}
  >{\centering\arraybackslash}p{(\columnwidth - 12\tabcolsep) * \real{0.1444}}
  >{\centering\arraybackslash}p{(\columnwidth - 12\tabcolsep) * \real{0.1444}}@{}}
\toprule\noalign{}
\begin{minipage}[b]{\linewidth}\centering
\end{minipage} & \begin{minipage}[b]{\linewidth}\centering
EM --\textgreater{} UE
\end{minipage} & \begin{minipage}[b]{\linewidth}\centering
EM --\textgreater{} OoLF
\end{minipage} & \begin{minipage}[b]{\linewidth}\centering
UE --\textgreater{} OoLF
\end{minipage} & \begin{minipage}[b]{\linewidth}\centering
UE --\textgreater{} EM
\end{minipage} & \begin{minipage}[b]{\linewidth}\centering
OoLF --\textgreater{} EM
\end{minipage} & \begin{minipage}[b]{\linewidth}\centering
OoLF --\textgreater{} UE
\end{minipage} \\
\midrule\noalign{}
\endhead
\bottomrule\noalign{}
\endlastfoot
Change in LF & NC & Dec & Dec & NC & Inc & Inc \\
Change in LFPR & NC & Dec & Dec & NC & Inc & Inc \\
Change in UER & Inc & Inc & Dec & Dec & Dec & Inc \\
\end{longtable}

\newpage{}

\section{CPI and Inflation Rate
(01/31)}\label{cpi-and-inflation-rate-0131}

\subsection{The Auction Game}\label{the-auction-game}

\subsubsection{Classroom Data}\label{classroom-data}

\begin{longtable}[]{@{}
  >{\raggedright\arraybackslash}p{(\columnwidth - 12\tabcolsep) * \real{0.0984}}
  >{\raggedright\arraybackslash}p{(\columnwidth - 12\tabcolsep) * \real{0.1557}}
  >{\raggedright\arraybackslash}p{(\columnwidth - 12\tabcolsep) * \real{0.1311}}
  >{\raggedright\arraybackslash}p{(\columnwidth - 12\tabcolsep) * \real{0.1557}}
  >{\raggedright\arraybackslash}p{(\columnwidth - 12\tabcolsep) * \real{0.1311}}
  >{\raggedright\arraybackslash}p{(\columnwidth - 12\tabcolsep) * \real{0.1557}}
  >{\raggedright\arraybackslash}p{(\columnwidth - 12\tabcolsep) * \real{0.1311}}@{}}
\toprule\noalign{}
\begin{minipage}[b]{\linewidth}\raggedright
Product
\end{minipage} & \begin{minipage}[b]{\linewidth}\raggedright
Round 1 Quantity
\end{minipage} & \begin{minipage}[b]{\linewidth}\raggedright
Round 1 Price
\end{minipage} & \begin{minipage}[b]{\linewidth}\raggedright
Round 2 Quantity
\end{minipage} & \begin{minipage}[b]{\linewidth}\raggedright
Round 2 Price
\end{minipage} & \begin{minipage}[b]{\linewidth}\raggedright
Round 3 Quantity
\end{minipage} & \begin{minipage}[b]{\linewidth}\raggedright
Round 3 Price
\end{minipage} \\
\midrule\noalign{}
\endhead
\bottomrule\noalign{}
\endlastfoot
Candy Bag & 1 & 4 & 2 & 20 & 3 & 145 \\
\end{longtable}

\begin{longtable}[]{@{}
  >{\raggedright\arraybackslash}p{(\columnwidth - 14\tabcolsep) * \real{0.0952}}
  >{\raggedright\arraybackslash}p{(\columnwidth - 14\tabcolsep) * \real{0.1333}}
  >{\raggedright\arraybackslash}p{(\columnwidth - 14\tabcolsep) * \real{0.1333}}
  >{\raggedright\arraybackslash}p{(\columnwidth - 14\tabcolsep) * \real{0.0952}}
  >{\raggedright\arraybackslash}p{(\columnwidth - 14\tabcolsep) * \real{0.1333}}
  >{\raggedright\arraybackslash}p{(\columnwidth - 14\tabcolsep) * \real{0.1429}}
  >{\raggedright\arraybackslash}p{(\columnwidth - 14\tabcolsep) * \real{0.0857}}
  >{\raggedright\arraybackslash}p{(\columnwidth - 14\tabcolsep) * \real{0.1238}}@{}}
\toprule\noalign{}
\begin{minipage}[b]{\linewidth}\raggedright
Rnd 1\\
Price\strut
\end{minipage} & \begin{minipage}[b]{\linewidth}\raggedright
Round 1\\
Remaining\\
Money\\
Supply\strut
\end{minipage} & \begin{minipage}[b]{\linewidth}\raggedright
Round 1\\
Additional\\
Money\\
Supply\strut
\end{minipage} & \begin{minipage}[b]{\linewidth}\raggedright
Rnd 2\\
Price\strut
\end{minipage} & \begin{minipage}[b]{\linewidth}\raggedright
Round 2\\
Remaining\\
Money\\
Supply\strut
\end{minipage} & \begin{minipage}[b]{\linewidth}\raggedright
Round 2\\
Additional\\
Money\\
Supply\strut
\end{minipage} & \begin{minipage}[b]{\linewidth}\raggedright
Rnd 3\\
Price\strut
\end{minipage} & \begin{minipage}[b]{\linewidth}\raggedright
Rnd 3\\
Remaining\\
Money\\
Supply\\
\strut
\end{minipage} \\
\midrule\noalign{}
\endhead
\bottomrule\noalign{}
\endlastfoot
10 & 90 & 10 & 20 & 80 & 20 & 30 & 70 \\
\end{longtable}

\subsubsection{Questions (I didn't finish these but I'll come back to
them
later)}\label{questions-i-didnt-finish-these-but-ill-come-back-to-them-later}

\begin{enumerate}
\def\labelenumi{\arabic{enumi}.}
\tightlist
\item
  What do the beans held by each person represent?
\end{enumerate}

Money

\begin{enumerate}
\def\labelenumi{\arabic{enumi}.}
\setcounter{enumi}{1}
\tightlist
\item
  What do all the beans in the room reperesent?
\end{enumerate}

All of the money in the economy

\begin{enumerate}
\def\labelenumi{\arabic{enumi}.}
\setcounter{enumi}{2}
\tightlist
\item
  What do you notice about the quantities across the 3 different rounds
  of the auction?
\end{enumerate}

Placeholder answer

\begin{enumerate}
\def\labelenumi{\arabic{enumi}.}
\setcounter{enumi}{3}
\tightlist
\item
  What do you notice about the prices across the 3 different rounds of
  the action?
\end{enumerate}

Placeholder answer

\begin{enumerate}
\def\labelenumi{\arabic{enumi}.}
\setcounter{enumi}{4}
\tightlist
\item
  With respect to your observations in \#4, why did this happen?
\end{enumerate}

Placeholder answer

\begin{enumerate}
\def\labelenumi{\arabic{enumi}.}
\setcounter{enumi}{5}
\tightlist
\item
  Does this simulation demonstrate price stability or price instability?
\end{enumerate}

Placeholder answer

\begin{enumerate}
\def\labelenumi{\arabic{enumi}.}
\setcounter{enumi}{6}
\tightlist
\item
  Who was hurt the most across the 3 rounds? The least?
\end{enumerate}

Placeholder answer

\begin{enumerate}
\def\labelenumi{\arabic{enumi}.}
\setcounter{enumi}{7}
\tightlist
\item
  Did the quality of the goods change across the rounds of gameplay? Was
  this in line with the tendency of prices?
\end{enumerate}

Placeholder answer

\begin{enumerate}
\def\labelenumi{\arabic{enumi}.}
\setcounter{enumi}{8}
\tightlist
\item
  Did quantity or price appear to have the greater percentage of change
  across the rounds?
\end{enumerate}

Placeholder answer

\begin{enumerate}
\def\labelenumi{\arabic{enumi}.}
\setcounter{enumi}{9}
\tightlist
\item
  Does the total spending across the 3 rounds reflect the change in
  production or consumption? What econmic measures are used to reflect
  this phenomenon?
\end{enumerate}

Placeholder answer

\begin{enumerate}
\def\labelenumi{\arabic{enumi}.}
\setcounter{enumi}{10}
\tightlist
\item
  Under what conditions did increasing the money supply (beans) cause
  inflation? Under what conditions did increasing the money supply
  (beans) not cause inflation?
\end{enumerate}

Placeholder answer

\subsection{CPI}\label{cpi}

The \textbf{CPI} (consumer price index) is a measure of the average
change over time in the prices paid by urban consumers for a market
basket of consumer goods and services.

\begin{itemize}
\tightlist
\item
  The 8 major groups of the CPI

  \begin{itemize}
  \tightlist
  \item
    Food and beverages
  \item
    Housing
  \item
    Apparel
  \item
    Transportation
  \item
    Medical care
  \item
    Recreation
  \item
    Education and communication
  \item
    Other goods and services
  \end{itemize}
\item
  It \textbf{does not include} life insurance, social security, or
  income taxes.
\item
  The core CPI is the CPI excluding food and energy prices.
\end{itemize}

\subsection{Calculating CPI}\label{calculating-cpi}

The formula for calculating the CPI is:

\(CPI = \frac{\Sigma (P_c \cdot Q_b)}{\Sigma (P_b \cdot Q_b)} \cdot 100\%\)

Where \(P_c\) is the current price, \(P_b\) is the base year price, and
\(Q_b\) is the quantity of the good/service.

In other words, the CPI is the \textbf{ratio} of the cost of the market
basket in the current year to the cost of the market basket in the base
year times 100\%.

\subsection{Inflation Rate}\label{inflation-rate}

The formula for calculating the inflation rate is:

\(\text{Inflation Rate} = \frac{\text{CPI}_{\text{c}} - \text{CPI}_{\text{b}}}{\text{CPI}_{\text{b}}} \cdot 100\%\)

\subsection{Main Idea: What is the difference between the CPI and the
GDP
deflator?}\label{main-idea-what-is-the-difference-between-the-cpi-and-the-gdp-deflator}

The \textbf{CPI} inflator is for \emph{consumers}, while the
\textbf{GDP} deflator is for \emph{producers}.

\section{The Business Cycle (02/02)}\label{the-business-cycle-0202}

\begin{figure}[H]

{\centering \includegraphics[width=1\textwidth,height=\textheight]{img/business-cycle.png}

}

\caption{The Business Cycle}

\end{figure}%

\section{Who Is Hurt/Benefits from (Unanticipated) Inflation?
(02/05)}\label{who-is-hurtbenefits-from-unanticipated-inflation-0205}

Technically, normal inflation is fine. That's because normal inflation
is expected and accounted for in the market. Unanticipated inflation is
the problem.

\subsection{\texorpdfstring{How to know who is hurt/helped by inflation
(\(\pi\))}{How to know who is hurt/helped by inflation (\textbackslash pi)}}\label{how-to-know-who-is-hurthelped-by-inflation-pi}

\subsubsection{\texorpdfstring{If
\(\text{Actual } π > \text{Expected } \pi\):}{If \textbackslash text\{Actual \} π \textgreater{} \textbackslash text\{Expected \} \textbackslash pi:}}\label{if-textactual-ux3c0-textexpected-pi}

\begin{itemize}
\tightlist
\item
  \textbf{Creditors/Lenders} are \emph{hurt} because they loan money at
  a \textbf{fixed rate} based on the expected interest rate and will
  therefore receive less than if they based it off of the actual
  interest rate
\item
  \textbf{Savers} are \emph{hurt} because their savings have less
  purchasing power
\item
  \textbf{Consumers with fixed incomes} are \emph{hurt} because they can
  only buy less with their fixed income
\item
  \textbf{Borrowers/Debtors} are \emph{helped} because they pay back
  their loans with money that has less purchasing power than when they
  borrowed it
\item
  \textbf{Firms with small resource costs} are \emph{helped} because
  product prices are rising faster than the costs of production,
  increasing their total revenue
\end{itemize}

\subsubsection{\texorpdfstring{If
\(\text{Actual } \pi > \text{Expected } \pi\)}{If \textbackslash text\{Actual \} \textbackslash pi \textgreater{} \textbackslash text\{Expected \} \textbackslash pi}}\label{if-textactual-pi-textexpected-pi}

Flip the above statements.

\subsubsection{Groups unaffected by
inflation}\label{groups-unaffected-by-inflation}

\begin{itemize}
\tightlist
\item
  Workers that have a \textbf{cost-of-living adjustment} (COLA) in their
  contracts (typically union workers)

  \begin{itemize}
  \tightlist
  \item
    \textbf{Social Security} \emph{does} have a COLA, but it's fairly
    minimal and not always enough to keep up with inflation.
  \end{itemize}
\item
  Pretty much anything else with an \textbf{adjustable rate}
\end{itemize}

\subsection{How to Calculate Real Income from
Inflation:}\label{how-to-calculate-real-income-from-inflation}

\begin{itemize}
\tightlist
\item
  \(\text{Real Income} = \frac{\text{Nominal Income}}{\text{CPI}}\cdot 100\)
\item
  \(\% \Delta \text{Real Wage} = \% \Delta \text{Nominal Wage} - \% \Delta \text{CPI}\)
\end{itemize}

\subsection{Why does inflation happen?}\label{why-does-inflation-happen}

There are three main theories to explain inflation:

\subsubsection{Quantity Theory}\label{quantity-theory}

\begin{itemize}
\tightlist
\item
  An increase in the supply of money drives prices up
\item
  This causes the value of currency to decrease and prices to rise by
  the same magnitude

  \begin{itemize}
  \tightlist
  \item
    For example, if the value of the dollar decreases by 50\%, prices
    will double to compensate, yielding 100\% inflation
  \end{itemize}
\item
  This theory can easily lead to hyperinflation (\(\pi \ge 200\%\) in
  \textless{} 1 yr.) if the money supply is increased too much
\end{itemize}

\subsubsection{Demand-Pull Theory}\label{demand-pull-theory}

\begin{itemize}
\tightlist
\item
  A spike in consumerism causes an increase in demand for goods and
  services, which causes prices to rise
\item
  Very common around the holiday season or Black Friday
\end{itemize}

\subsubsection{Cost-Push Theory}\label{cost-push-theory}

\begin{itemize}
\tightlist
\item
  Increased resource costs in the factor market causes firms to scale
  back production and raise prices
\item
  The \textbf{wage-price spiral} is a common example of cost-push
  inflation

  \begin{itemize}
  \tightlist
  \item
    Workers demand higher wages to keep up with inflation, which causes
    firms to raise prices, which causes workers to demand higher wages,
    and so on
  \end{itemize}
\item
  Can also be caused by an overall decrease in the supply of resources
  because of a natural disaster or war
\end{itemize}

\subsection{Costs of Inflation}\label{costs-of-inflation}

\begin{itemize}
\tightlist
\item
  Menu costs result from a firm having to change prices.
\item
  Shoe leather costs refer to the time and effort that people spend to
  counteract the effects of inflation.
\end{itemize}

\newpage{}

\section{Introduction to Aggregate Demand/Supply
(02/12)}\label{introduction-to-aggregate-demandsupply-0212}

\subsection{The Aggregate Models
(Graphs)}\label{the-aggregate-models-graphs}

\includegraphics{img/agg-models.png}

\begin{longtable}[]{@{}
  >{\raggedright\arraybackslash}p{(\columnwidth - 2\tabcolsep) * \real{0.4854}}
  >{\raggedright\arraybackslash}p{(\columnwidth - 2\tabcolsep) * \real{0.5146}}@{}}
\toprule\noalign{}
\begin{minipage}[b]{\linewidth}\raggedright
\begin{verbatim}
    Supply and Demand Model (Graph)
\end{verbatim}
\end{minipage} & \begin{minipage}[b]{\linewidth}\raggedright
\begin{verbatim}
Aggregate Supply and Demand Model (Graph)
\end{verbatim}
\end{minipage} \\
\midrule\noalign{}
\endhead
\bottomrule\noalign{}
\endlastfoot
\begin{minipage}[t]{\linewidth}\raggedright
\textbf{What are the two laws?}\\
Law of Supply and Law of Demand\strut
\end{minipage} & \begin{minipage}[t]{\linewidth}\raggedright
\textbf{What does the term ``aggregate'' mean?}\\
Sum\strut
\end{minipage} \\
\includegraphics[width=0.5\textwidth,height=\textheight]{img/supply-demand.png}
&
\includegraphics[width=0.5\textwidth,height=\textheight]{img/agg-model.png} \\
\begin{minipage}[t]{\linewidth}\raggedright
\textbf{What/Who does the demand curve represent?}\\
Inverse relationship between \(P\) and \(Q_d\)\strut
\end{minipage} & \begin{minipage}[t]{\linewidth}\raggedright
\textbf{What/Who does the AD curve represent?}\\
Aggregate Expenditure = Aggregate Income\\
\(C+I+G+N_x\) = \(W+R+I+P\)\strut
\end{minipage} \\
\begin{minipage}[t]{\linewidth}\raggedright
\textbf{What/Who does the supply curve represent?}\\
Direct relation between \(P\) and \(Q_s\)\strut
\end{minipage} & \begin{minipage}[t]{\linewidth}\raggedright
\textbf{What/Who does the AS curve represent?}\\
Aggregate production\strut
\end{minipage} \\
\begin{minipage}[t]{\linewidth}\raggedright
\textbf{Reason for Change in Quantity Demanded}\\
Price ONLY!\strut
\end{minipage} & \begin{minipage}[t]{\linewidth}\raggedright
\textbf{Reasons for Change in Quantity of AD}\\
Price level only\strut
\end{minipage} \\
\begin{minipage}[t]{\linewidth}\raggedright
\textbf{Reason for Change in Demand}\\
BITER\strut
\end{minipage} & \begin{minipage}[t]{\linewidth}\raggedright
\textbf{Reasons for Change of AD}\\
\(C+I+G+N_x = W+R+I+P\)\strut
\end{minipage} \\
\begin{minipage}[t]{\linewidth}\raggedright
\textbf{Reason for Change in Quantity Supplied}\\
Price ONLY!\strut
\end{minipage} & \begin{minipage}[t]{\linewidth}\raggedright
\textbf{Reasons for Change in Quantity of AS}\\
Price level ONLY!\strut
\end{minipage} \\
\begin{minipage}[t]{\linewidth}\raggedright
\textbf{Reason for Change in Supply}\\
TONERS\strut
\end{minipage} & \begin{minipage}[t]{\linewidth}\raggedright
\textbf{Reasons for Change in AS}\\
IRAP:\\
Inflationary Expectations\\
Resource Cost and Availability\\
Acts of Mother Nature and Government Regulations\\
Productivity\strut
\end{minipage} \\
\end{longtable}

\subsection{Why is aggregate demand downward
sloping?}\label{why-is-aggregate-demand-downward-sloping}

All of these effects demonstrate that \(PL \propto \frac{1}{Q_d}\): ~ -
\textbf{Real Wealth Effect}: \(PL\) decreases, purchasing power
increases, consumption increases - \textbf{Interest Rate Effect}: \(PL\)
decreases, interest rate decreases, investment increases -
\textbf{Exchange Rate Effect}: \(PL\) relative to another country
decreases, exports increase, RGDP increases

\subsection{Why is short run aggregate supply upward
sloping?}\label{why-is-short-run-aggregate-supply-upward-sloping}

All of these effects demonstrate that \(PL \propto Q_s\):\\
- \textbf{Sticky Wage Theory (wage rigidity)}: Wages adjust slowly to
\(PL\), so when \(PL\) increases, the real wage falls and employment
increases - \textbf{Sticky Price Theory (price rigidity)}: \(PL\)
adjusts slowly to economic changes, and businesses don't want to incur
menu costs often, so they increase production instead -
\textbf{Misperception Theory}: For a single firm, \(PL\) decreasing
increases buyer consumption, so they increase production; however, for
the economy as a whole, this is not the case and they should have
increased prices instead

\subsubsection{Real Wealth Sequence}\label{real-wealth-sequence}

PL

\section{Shifters in AD/AS (02/13)}\label{shifters-in-adas-0213}

\subsection{AD/AS You Decide}\label{adas-you-decide}

\begin{longtable}[]{@{}
  >{\raggedright\arraybackslash}p{(\columnwidth - 6\tabcolsep) * \real{0.3566}}
  >{\raggedright\arraybackslash}p{(\columnwidth - 6\tabcolsep) * \real{0.3876}}
  >{\raggedright\arraybackslash}p{(\columnwidth - 6\tabcolsep) * \real{0.1240}}
  >{\raggedright\arraybackslash}p{(\columnwidth - 6\tabcolsep) * \real{0.1163}}@{}}
\toprule\noalign{}
\begin{minipage}[b]{\linewidth}\raggedright
\begin{verbatim}
           Case Scenario
\end{verbatim}
\end{minipage} & \begin{minipage}[b]{\linewidth}\raggedright
Draw a correctly labeled graph indicating the\\
change in AD or AS\strut
\end{minipage} & \begin{minipage}[b]{\linewidth}\raggedright
Increase or\\
Decrease of\\
PL and RGDP\strut
\end{minipage} & \begin{minipage}[b]{\linewidth}\raggedright
Which Gap or Economic\\
Growth?\strut
\end{minipage} \\
\midrule\noalign{}
\endhead
\bottomrule\noalign{}
\endlastfoot
\begin{minipage}[t]{\linewidth}\raggedright
Tar sand/Oil sand in Canada and\\
Nebraska allows the US to gain energy\\
independence from the Middle East\strut
\end{minipage} & & \begin{minipage}[t]{\linewidth}\raggedright
PL Inc.~\\
RGDP Inc.\strut
\end{minipage} & \\
\begin{minipage}[t]{\linewidth}\raggedright
More seniors eat waffles topped with\\
whip cream at their senior breakfasts all\\
pver the US\strut
\end{minipage} & & \begin{minipage}[t]{\linewidth}\raggedright
PL Inc.~\\
RGDP Inc.\strut
\end{minipage} & \\
\begin{minipage}[t]{\linewidth}\raggedright
A squirrel invasion destroys the harvest\\
of Walnuts in California\strut
\end{minipage} & & \begin{minipage}[t]{\linewidth}\raggedright
PL Inc.~\\
RGDP Inc.\strut
\end{minipage} & \\
\begin{minipage}[t]{\linewidth}\raggedright
CIA spends a greater portion of its\\
budget on drone strikes\strut
\end{minipage} & & \begin{minipage}[t]{\linewidth}\raggedright
PL Inc.~\\
RGDP Inc.\strut
\end{minipage} & \\
\begin{minipage}[t]{\linewidth}\raggedright
More seniors eat waffles topped with\\
whip cream at their senior breakfasts all\\
pver the US\strut
\end{minipage} & & \begin{minipage}[t]{\linewidth}\raggedright
PL Inc.~\\
RGDP Inc.\strut
\end{minipage} & \\
\end{longtable}

\subsection{AD vs AS Shifters}\label{ad-vs-as-shifters}

\begin{longtable}[]{@{}
  >{\raggedright\arraybackslash}p{(\columnwidth - 6\tabcolsep) * \real{0.3564}}
  >{\raggedright\arraybackslash}p{(\columnwidth - 6\tabcolsep) * \real{0.1584}}
  >{\raggedright\arraybackslash}p{(\columnwidth - 6\tabcolsep) * \real{0.3564}}
  >{\raggedright\arraybackslash}p{(\columnwidth - 6\tabcolsep) * \real{0.1089}}@{}}
\toprule\noalign{}
\begin{minipage}[b]{\linewidth}\raggedright
\begin{verbatim}
        Scenario
\end{verbatim}
\end{minipage} & \begin{minipage}[b]{\linewidth}\raggedright
Shifter
\end{minipage} & \begin{minipage}[b]{\linewidth}\raggedright
\begin{verbatim}
          Graph
\end{verbatim}
\end{minipage} & \begin{minipage}[b]{\linewidth}\raggedright
\end{minipage} \\
\midrule\noalign{}
\endhead
\bottomrule\noalign{}
\endlastfoot
Congress passes a tax increase & Taxes &
\includegraphics[width=0.5\textwidth,height=\textheight]{img/agg-model.png}
Price Dec. RGDP Dec. & AD Left \\
\begin{minipage}[t]{\linewidth}\raggedright
The Government is increasing\\
its spending on infrastructure\strut
\end{minipage} & \begin{minipage}[t]{\linewidth}\raggedright
Government\\
Spending\strut
\end{minipage} &
\includegraphics[width=0.5\textwidth,height=\textheight]{img/agg-model.png}
Price Inc RGDP Inc & AD Right \\
\begin{minipage}[t]{\linewidth}\raggedright
Microchip shortages are hitting\\
the car industry\strut
\end{minipage} & \begin{minipage}[t]{\linewidth}\raggedright
Supply\\
Shortage\strut
\end{minipage} &
\includegraphics[width=0.5\textwidth,height=\textheight]{img/agg-model.png}
Price Inc. RGDP Dec. & AS Left \\
\begin{minipage}[t]{\linewidth}\raggedright
Producers believe that future\\
prices are expected to rise\strut
\end{minipage} & Expectations &
\includegraphics[width=0.5\textwidth,height=\textheight]{img/agg-model.png}
Price Inc. RGDP Dec. & AS Right \\
\begin{minipage}[t]{\linewidth}\raggedright
The cost of plastic for toy\\
manufacturers has fallen\strut
\end{minipage} & \begin{minipage}[t]{\linewidth}\raggedright
Costs of\\
Production\strut
\end{minipage} &
\includegraphics[width=0.5\textwidth,height=\textheight]{img/agg-model.png}
Price Dec. RGDP Inc. & AS Left \\
\begin{minipage}[t]{\linewidth}\raggedright
Consumers are worried about\\
the future of the US economy\strut
\end{minipage} & Expectations &
\includegraphics[width=0.5\textwidth,height=\textheight]{img/agg-model.png}
Price Inc RGDP Inc & AD Left \\
\begin{minipage}[t]{\linewidth}\raggedright
New regulations make less\\
forest available for timber mills\strut
\end{minipage} & \begin{minipage}[t]{\linewidth}\raggedright
Acts of the\\
Government\strut
\end{minipage} &
\includegraphics[width=0.5\textwidth,height=\textheight]{img/agg-model.png}
Price Inc. RGDP Dec. & AS Left \\
\begin{minipage}[t]{\linewidth}\raggedright
Firms increase their spending\\
on education for employees\strut
\end{minipage} & Investment &
\includegraphics[width=0.5\textwidth,height=\textheight]{img/agg-model.png}
Price Inc. RGDP Inc. & AD Right \\
\end{longtable}

\subsection{PPF vs LRAS model}\label{ppf-vs-lras-model}

\begin{longtable}[]{@{}
  >{\raggedright\arraybackslash}p{(\columnwidth - 2\tabcolsep) * \real{0.5321}}
  >{\raggedright\arraybackslash}p{(\columnwidth - 2\tabcolsep) * \real{0.4679}}@{}}
\toprule\noalign{}
\begin{minipage}[b]{\linewidth}\raggedright
\begin{verbatim}
        Production Possibility Frontier
\end{verbatim}
\end{minipage} & \begin{minipage}[b]{\linewidth}\raggedright
\begin{verbatim}
    Long-Run Aggregate Supply Model
\end{verbatim}
\end{minipage} \\
\midrule\noalign{}
\endhead
\bottomrule\noalign{}
\endlastfoot
\begin{minipage}[t]{\linewidth}\raggedright
\textbf{What does the term ``production possibility'' mean?}\\
The ideal production for a given resource.\strut
\end{minipage} & \begin{minipage}[t]{\linewidth}\raggedright
\textbf{What does the term ``potential'' mean?}\\
a\strut
\end{minipage} \\
\includegraphics[width=0.5\textwidth,height=\textheight]{img/agg-model.png}
&
\includegraphics[width=0.5\textwidth,height=\textheight]{img/agg-model.png} \\
\begin{minipage}[t]{\linewidth}\raggedright
\textbf{What/Who does the PPF represent?}\\
a\strut
\end{minipage} & \begin{minipage}[t]{\linewidth}\raggedright
\textbf{What/Who does the LRAS curve represent?}\\
All producers in LR\strut
\end{minipage} \\
\begin{minipage}[t]{\linewidth}\raggedright
\textbf{What does being under the curve represent?}\\
Inefficient\strut
\end{minipage} & \begin{minipage}[t]{\linewidth}\raggedright
\textbf{What does beign below the curve represent?}\\
Below full employment (negative output gap)\strut
\end{minipage} \\
\begin{minipage}[t]{\linewidth}\raggedright
\textbf{What does being on the curve represent?}\\
Efficient\strut
\end{minipage} & \begin{minipage}[t]{\linewidth}\raggedright
\textbf{What does being at the curve represent?}\\
Zero output at AD\strut
\end{minipage} \\
\begin{minipage}[t]{\linewidth}\raggedright
\textbf{What does being above the curve represent?}\\
Unattainable in the long run\strut
\end{minipage} & \begin{minipage}[t]{\linewidth}\raggedright
\textbf{What does being above the curve represent?}\\
Above full employment (positive output gap)\strut
\end{minipage} \\
\begin{minipage}[t]{\linewidth}\raggedright
\textbf{What are the shifters?}\\
1. Tech\\
2. Resources\\
3. Productivity\strut
\end{minipage} & \begin{minipage}[t]{\linewidth}\raggedright
\textbf{What are the shifters?}\\
1. Tech 2. Resources/Capital stock\\
3. Productivity\strut
\end{minipage} \\
\end{longtable}

\subsection{The four aggregate
positions}\label{the-four-aggregate-positions}

\begin{longtable}[]{@{}
  >{\raggedright\arraybackslash}p{(\columnwidth - 0\tabcolsep) * \real{1.0076}}@{}}
\toprule\noalign{}
\endhead
\bottomrule\noalign{}
\endlastfoot
\begin{minipage}[t]{\linewidth}\raggedright
\begin{verbatim}
           **Inflationary Gap** \               |               **Recessionary Gap**\                |
      Type of inflation: Demand-Pull \          |                                                    |
\end{verbatim}

\(RGDP_{\text{actual}} > RGDP_{\text{potential}}\) ~\textbar{}
\(RGDP_{\text{actual}} < RGDP_{\text{potential}}\) ~\textbar{} Effect on
UE: Decrease \textbar{} Effect on UE: Increase \textbar{}
====================================================+====================================================+
Graph ~ \textbar{} Graph ~ \textbar{}
\includegraphics[width=0.5\textwidth,height=\textheight]{img/agg-model.png}
\textbar{}
\includegraphics[width=0.5\textwidth,height=\textheight]{img/agg-model.png}
\end{minipage} \\
\textbf{Period of SR Economic Growth} ~ \textbar{} \textbf{Period of
Stagflation} ~ \textbar{} \textbar{} Type of inflation: Cost-Push ~
\textbar{} \(RGDP_{\text{actual}} > RGDP_{\text{potential}}\)
~\textbar{} \(RGDP_{\text{actual}} < RGDP_{\text{potential}}\)
~\textbar{} Effect on UE: Decrease \textbar{} Effect on UE: Increase
\textbar{}
====================================================+====================================================+
Graph ~ \textbar{} Graph ~ \textbar{}
\includegraphics[width=0.5\textwidth,height=\textheight]{img/agg-model.png}
\textbar{}
\includegraphics[width=0.5\textwidth,height=\textheight]{img/agg-model.png} \\
\end{longtable}

\newpage{}

\section{LR-Adjustments (02/20)}\label{lr-adjustments-0220}

\section{Propensity and the Multiplier
(02/23)}\label{propensity-and-the-multiplier-0223}

\section{Fiscal Policy (02/26)}\label{fiscal-policy-0226}

\begin{longtable}[]{@{}
  >{\raggedright\arraybackslash}p{(\columnwidth - 2\tabcolsep) * \real{0.3790}}
  >{\raggedright\arraybackslash}p{(\columnwidth - 2\tabcolsep) * \real{0.6210}}@{}}
\toprule\noalign{}
\begin{minipage}[b]{\linewidth}\raggedright
Expansionary Fiscal Policy
\end{minipage} & \begin{minipage}[b]{\linewidth}\raggedright
Contractionary Fiscal Policy
\end{minipage} \\
\begin{minipage}[b]{\linewidth}\raggedright
Applied durnig which phase of the business\\
cycle:\\
\strut
\end{minipage} & \begin{minipage}[b]{\linewidth}\raggedright
Applied during which phase of the business\\
cycle:\\
\(G \downarrow \therefore AD \downarrow\)\\
\(T_{\text{bus}} \uparrow \therefore I \downarrow \therefore AD \downarrow\)\strut
\end{minipage} \\
\midrule\noalign{}
\endhead
\bottomrule\noalign{}
\endlastfoot
Tools: & Tools: \\
Effect on AD (Graph): & Effect on AD (Graph): \\
\end{longtable}

\newpage{}

\subsection{Government Multipliers}\label{government-multipliers}

The federal government gives you \$100.

The economy's MPC = 0.9 and MPS = 0.1.

\begin{longtable}[]{@{}
  >{\raggedright\arraybackslash}p{(\columnwidth - 4\tabcolsep) * \real{0.2500}}
  >{\raggedright\arraybackslash}p{(\columnwidth - 4\tabcolsep) * \real{0.1731}}
  >{\raggedright\arraybackslash}p{(\columnwidth - 4\tabcolsep) * \real{0.5673}}@{}}
\toprule\noalign{}
\endhead
\bottomrule\noalign{}
\endlastfoot
Multiplier & Formula & Multiplier Effect \\
\begin{minipage}[t]{\linewidth}\raggedright
Government Spending\\
Multiplier for Fiscal\\
Policy\strut
\end{minipage} & \(\frac{1}{MPS}\) &
\(\Delta G \cdot \text{Mult} = \Delta \text{Real Output}\) \\
\end{longtable}

From the government's fiscal spending, the total additional spending in
our economy is \$1000. What should the government do to close the gap?

\begin{longtable}[]{@{}
  >{\raggedright\arraybackslash}p{(\columnwidth - 10\tabcolsep) * \real{0.1477}}
  >{\raggedright\arraybackslash}p{(\columnwidth - 10\tabcolsep) * \real{0.1477}}
  >{\raggedright\arraybackslash}p{(\columnwidth - 10\tabcolsep) * \real{0.2273}}
  >{\raggedright\arraybackslash}p{(\columnwidth - 10\tabcolsep) * \real{0.1705}}
  >{\raggedright\arraybackslash}p{(\columnwidth - 10\tabcolsep) * \real{0.2500}}
  >{\raggedright\arraybackslash}p{(\columnwidth - 10\tabcolsep) * \real{0.0114}}@{}}
\toprule\noalign{}
\endhead
\bottomrule\noalign{}
\endlastfoot
Propensity & Multiplier & Aggregate Economy & Size of Gap & Government
Spending & \\
MPS = 0.2 & 5 & Recessionary Gap &
\begin{minipage}[t]{\linewidth}\raggedright
-\$100\\
million\strut
\end{minipage} & Increase \$ & \\
\multicolumn{6}{@{}>{\raggedright\arraybackslash}p{(\columnwidth - 10\tabcolsep) * \real{0.9545} + 10\tabcolsep}@{}}{%
MPS = 0.5 \textbar{} 2 \textbar{} Inflationary Gap \textbar{} \$50
million \textbar{} Decrease \$} \\
MPS = 0.33 & 3 & Stagflation &
\begin{minipage}[t]{\linewidth}\raggedright
-\$200\\
million\strut
\end{minipage} & N/A & \\
\end{longtable}

Now, the federal government reduces taxes by \$100 while allowing the
economy to keep what is already has.

The economy's MPC = 0.9 and MPS = 0.1.

\begin{longtable}[]{@{}
  >{\raggedright\arraybackslash}p{(\columnwidth - 4\tabcolsep) * \real{0.2703}}
  >{\raggedright\arraybackslash}p{(\columnwidth - 4\tabcolsep) * \real{0.1892}}
  >{\raggedright\arraybackslash}p{(\columnwidth - 4\tabcolsep) * \real{0.5315}}@{}}
\toprule\noalign{}
\endhead
\bottomrule\noalign{}
\endlastfoot
Multiplier & Formula & Multiplier Effect \\
\begin{minipage}[t]{\linewidth}\raggedright
Tax Multiplier for Fiscal\\
Policy\strut
\end{minipage} & \(-\frac{MPC}{MPS}\) &
\(\Delta T \cdot \text{Mult} = \Delta \text{Real Output}\)
\(-100 \cdot \frac{-0.9}{-0.1} = \$900\) \\
\end{longtable}

The total spending in the economy is \$900. What should the government
do to cloes the gap?

\begin{longtable}[]{@{}
  >{\raggedright\arraybackslash}p{(\columnwidth - 8\tabcolsep) * \real{0.1512}}
  >{\raggedright\arraybackslash}p{(\columnwidth - 8\tabcolsep) * \real{0.1512}}
  >{\raggedright\arraybackslash}p{(\columnwidth - 8\tabcolsep) * \real{0.2326}}
  >{\raggedright\arraybackslash}p{(\columnwidth - 8\tabcolsep) * \real{0.1744}}
  >{\raggedright\arraybackslash}p{(\columnwidth - 8\tabcolsep) * \real{0.2558}}@{}}
\toprule\noalign{}
\endhead
\bottomrule\noalign{}
\endlastfoot
Propensity & Multiplier & Aggregate Economy & Size of Gap & Government
Spending \\
MPS = 0.2 & -4 & Recessionary Gap &
\begin{minipage}[t]{\linewidth}\raggedright
-\$100\\
million\strut
\end{minipage} & Increase \$ \\
MS = 0.5 & -1 & Inflationary Gap & \$50 million & Decrease \$ \\
MPS = 0.33 & -2 & Stagflation &
\begin{minipage}[t]{\linewidth}\raggedright
-\$200\\
million\strut
\end{minipage} & N/A \\
\end{longtable}

ALSO do nothing because the economy is still in Stagflation.

\newpage{}

\subsection{Balanced Budget
Multiplier}\label{balanced-budget-multiplier}

Goals of fiscal policy:

\begin{enumerate}
\def\labelenumi{\arabic{enumi}.}
\tightlist
\item
  No surplus spending
\item
  No deficit spending
\item
  No change to our national debt
\end{enumerate}

Therefore\ldots{}

\begin{longtable}[]{@{}
  >{\raggedright\arraybackslash}p{(\columnwidth - 2\tabcolsep) * \real{0.5067}}
  >{\raggedright\arraybackslash}p{(\columnwidth - 2\tabcolsep) * \real{0.4933}}@{}}
\toprule\noalign{}
\endhead
\bottomrule\noalign{}
\endlastfoot
Spending Increase by \$100 \(\Longrightarrow\) \(\downarrow \$400\) &
Taxes \(\uparrow \$100\)\Longleftarrow\$ Decrease by \$400 \\
\end{longtable}

\newpage{}

\section{Money and Interest Rates
(03/06)}\label{money-and-interest-rates-0306}

\subsection{\texorpdfstring{Notetaker for
\href{https://www.youtube.com/watch?v=9ZJUkSePNLw}{this video} on
money}{Notetaker for this video on money}}\label{notetaker-for-this-video-on-money}

\begin{enumerate}
\def\labelenumi{\arabic{enumi}.}
\tightlist
\item
  Before money, what was used to facilitate exchanges?
\end{enumerate}

\textbf{Bartering}

\begin{enumerate}
\def\labelenumi{\arabic{enumi}.}
\setcounter{enumi}{1}
\tightlist
\item
  Name and describe the 3 functions of money.
\end{enumerate}

\begin{itemize}
\tightlist
\item
  \textbf{Medium of Exchange}: Money facilitates exchanges.
\item
  \textbf{Unit of Account}: Money measures the value of items.
\item
  \textbf{Store of Value}: Money can be saved and used later.
\end{itemize}

\begin{enumerate}
\def\labelenumi{\arabic{enumi}.}
\setcounter{enumi}{2}
\tightlist
\item
  Define the following types of money:
\end{enumerate}

\begin{itemize}
\tightlist
\item
  \textbf{Commodity}

  \begin{itemize}
  \tightlist
  \item
    Mediums of exchange with intrinsic value
  \end{itemize}
\item
  \textbf{Representative}

  \begin{itemize}
  \tightlist
  \item
    Each bill has a commodity that gives it value
  \end{itemize}
\item
  \textbf{Fiat}

  \begin{itemize}
  \tightlist
  \item
    Currency that gets its value from what it will buy
  \end{itemize}
\end{itemize}

\begin{enumerate}
\def\labelenumi{\arabic{enumi}.}
\setcounter{enumi}{3}
\tightlist
\item
  What are the measures of money?
\end{enumerate}

\begin{itemize}
\tightlist
\item
  \textbf{M0 (Monetary Base):} Currency in bank reserves/circulation
\item
  \textbf{M1 (Money)}: Currency in circulation, checkable deposits, and
  savings deposits
\item
  \textbf{M2 (Money and Near Money)}: Money not immediately available as
  a medium of exchange but can be converted to a medium of exchange (M1
  + small time deposits and money market mutual funds)
\end{itemize}

\includegraphics[width=0.7\textwidth,height=\textheight]{img/m0-m1-venn.png}

\subsection{\texorpdfstring{Refresher on interest rates + Fisher Formula
(\href{https://www.youtube.com/watch?v=KfPGfP97NBA}{video
notes})}{Refresher on interest rates + Fisher Formula (video notes)}}\label{refresher-on-interest-rates-fisher-formula-video-notes}

\subsubsection{The variables}\label{the-variables}

\begin{itemize}
\tightlist
\item
  \(i\): Nominal interest rate

  \begin{itemize}
  \tightlist
  \item
    Interest banks charge for loans
  \item
    Not adjusted for inflation
  \end{itemize}
\item
  \(r\) Real interest rate

  \begin{itemize}
  \tightlist
  \item
    Inflation-adjusted interest that banks charge
  \end{itemize}
\item
  \(\pi\): Inflation rate

  \begin{itemize}
  \tightlist
  \item
    Rate at which prices are rising/falling
  \item
    Measured by the CPI or GDP deflator
  \end{itemize}
\end{itemize}

\subsubsection{The Fisher Formula}\label{the-fisher-formula}

\(i \approx r + \pi\)

This means that banks will charge the real interest they desire plus the
expected inflation rate.

\subsubsection{Expected vs.~Actual
Inflation}\label{expected-vs.-actual-inflation}

\begin{itemize}
\tightlist
\item
  Actual inflation impacts the real interest rate because
  \(r \approx i - \pi\).

  \begin{itemize}
  \tightlist
  \item
    This means that \textbf{higher than expected inflation} \emph{helps}
    borrowers and \emph{hurts} lenders, and vice-versa
  \end{itemize}
\end{itemize}

\subsubsection{Other applications}\label{other-applications}

The Fisher Formula can be mirrored for changes in nominal/real wages and
GDP:

\begin{itemize}
\tightlist
\item
  GDP changes:
  \(\% \Delta \text{rGDP} \approx \% \Delta \text{nGDP} - \pi\)
\item
  Wage changes:
  \(\% \Delta \text{rW} \approx \% \Delta \text{nW} - \pi\)
\end{itemize}



\end{document}
